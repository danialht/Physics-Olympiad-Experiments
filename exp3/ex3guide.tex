\documentclass{article}

\usepackage{amsmath}
\usepackage{graphicx}
\usepackage{enumitem}
\usepackage{amsfonts}

\begin{document}

\begin{center}
    \huge Experiment 3: The Fall\\
    \Large Program Manual\\
    \small Developed by Mohammad Amin Haghjoo \\
    \small February 2025
\end{center}



In this experiment a hypothetical ball of mass $m$ is released from a height of $h$ in the presence of gravity $g$ and linear air drag.
\begin{equation}
    \textbf{F}_{\text{drag}}=-\alpha \textbf{v}
\end{equation}
The provided \texttt{.exe} file will input the height and mass from you and return the time that the ball hits the ground. Your task is to find the values of $g$ and $\alpha$.
\begin{center}
    \includegraphics[width = 0.75 \textwidth , height = 4 cm]{fig1.png} \\
    Fig 1, Program Interface
\end{center}
The times given to you include random error; therefore, standard statistical methods are required here.
\begin{center}
    \huge Program Guide
\end{center}
Notes :
\begin{enumerate}
    \item ALWAYS enter data in the requested format, DO NOT enter characters when numbers are asked for.
    \item The program keeps running until you shut it down by using the \texttt{EXIT} command when prompted.
    \item If your PC tells you that the program is from an unknown publisher and gives a warning, ignore it. There is no matter of concern.
    \item Perform your calculations in Microsoft Excel.
    \item Note that you can copy numbers from the program; to do so, select that part and use \texttt{ctrl + c} to copy it. You can also use \texttt{ctrl + v} to paste.
    \item To close the program by other means, use \texttt{ctrl + c} when no text is selected, closing the window is also possible.
\end{enumerate}

\vspace{0.4 cm}

\begin{center}
    \huge The Problem
\end{center}

\vspace{0.4 cm}

\begin{enumerate}[label = \alph*]
    \item Derive an equation relating the quantities $g,\ h,\ t, \ \gamma = \frac{\alpha}{m}$.
    \item Construct a table to document your data points and their errors, and any other quantities you may have defined.
    \item From your data and regressional analysis, find the values of $\alpha \ \text{and} \ g$.
\end{enumerate}

\begin{center}
    \huge Mathematical Reference
\end{center}
\vspace{0.5 cm}
Integrals :
\begin{align}
    \int e^x \ dx &=e^x+C& \ &C \in \mathbb{R} \\
    \int x^n \ dx &=\frac{x^{n+1}}{n+1}+C& \ &C \in \mathbb{R} \ \text{and} \ n\neq -1
\end{align}

Good Luck! \vspace{5 mm} -Mohammad Amin Haghjoo.

\end{document}